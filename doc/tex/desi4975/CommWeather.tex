\documentclass[12pt]{article}

\usepackage{amsmath}
\usepackage{graphicx}
\usepackage{hyperref}
\usepackage{cite}
\usepackage[margin=1.2in]{geometry}

\providecommand{\eqn}[1]{eqn.~(\ref{eqn:#1})}
\providecommand{\tab}[1]{Table~\ref{tab:#1}}
\providecommand{\fig}[1]{Figure~\ref{fig:#1}}

\title{Observing Conditions\\
during DESI Commissioning\\
\vspace{5mm}{\large\bf DESI-doc-4975-v1}}
\author{Bela Abolfathi and David Kirkby}

\begin{document}
\maketitle

\noindent {\bf Version History:}
\begin{itemize}
    \item v1: Dome open fraction from the CI run (Apr-May 2019).
\end{itemize}

\section{Introduction}

This note documents the observing conditions during the DESI Commissioning Run at the Mayall, and compares them
with the forecast model used for the survey margin estimates in DESI-4355~\cite{desi4355}.

The periods of continuous observing during commissioning that we consider are:
\begin{itemize}
\item Apr 1 -- May 5, 2019: Commissioning Instrument.
\item May 16 -- June 2, 2019: Commissioning Instrument.
\end{itemize}

The latex source for this note is maintained in {\tt /doc/tex/desi4975/} of the {\tt desimodel} package on github,
with an accompanying juypter notebook in {\tt /doc/nb/CommWeather.ipynb}. This version of the document corresponds to
version {\tt 0.9.10} of the {\tt desimodel} package.

\section{Methodology}

This section describes the methodology used to estimate the following observing conditions: dome-closed fraction,
delivered image quality, and sky transparency.  Details on the forecast models used for these quantities are in
DESI-3087~\cite{desi3087} and DESI-3577~\cite{desi3577}.

\subsection{Dome-Closed Fraction}

The dome-closed fraction is defined as the fraction of time the dome was closed compared to how long it could have been open if  observing conditions were favorable. Since information regarding the precise time the dome was opened, closed or re-opened was not readily available, the fraction of time the dome was open for observing was pieced together from the nightly observing logs. If the time of the opening of the dome was recorded in the observing log, this was used as the `start of the night.' Otherwise the start of the night was assumed to be the time of the first on-sky science exposure, which usually corresponded to astronomical twilight. The `end of the night' was the time the dome was closed as noted in the log, if available. If not, it was assumed to be the time of the last on-sky science exposure if it was taken close to the start of astronomical twilight. The amount of time the dome was closed during the night due to bad weather was gleaned from reading each night log individually and estimated either from what was written in the log or when on-sky exposures began and ended, again, depending on the information that was available. 

\subsection{Delivered Image Quality}

To be completed in a future version of this document.

\subsection{Sky Transparency}

To be completed in a future version of this document.

\section{Results}

\subsection{Dome-Closed Fraction}

...

\subsection{Delivered Image Quality}

To be completed in a future version of this document.

\subsection{Sky Transparency}

To be completed in a future version of this document.

\bibliographystyle{unsrt}
\bibliography{CommWeather}

\end{document}
